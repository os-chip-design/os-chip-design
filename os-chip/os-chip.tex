\documentclass[conference]{IEEEtran}


\usepackage{cite}
\usepackage{pslatex} % -- times instead of computer modern, especially for the plain article class
\usepackage[colorlinks=false,bookmarks=false]{hyperref}
\usepackage{booktabs}
\usepackage{graphicx}
\usepackage{xcolor}
\usepackage{multirow}
\usepackage{comment}
\usepackage{listings}
%\usepackage{flushend} % even out the last page, but use only at the end when there is a bibliography
%\usepackage{minted}		% For inserting code
%\setminted[systemverilog]{
%	tabsize=3
%}
%\setminted[C]{
%	tabsize=3,
%	breaklines
%}
%\setminted[scala]{
%	tabsize=3,
%	breaklines
%}
\usepackage{xspace}		% For using \SV with trailing spaces
\usepackage{cleveref}	% Needed for correctly referencing listings
\usepackage{subfig}

\newcommand{\code}[1]{{\small{\texttt{#1}}}}
\newcommand{\SV}{SystemVerilog\xspace}


% fatter TT font
\renewcommand*\ttdefault{txtt}
% another TT, suggested by Alex
% \usepackage{inconsolata}
% \usepackage[T1]{fontenc} % needed as well?

%\newcommand{\todo}[1]{{\emph{TODO: #1}}}
\newcommand{\todo}[1]{{\color{olive} TODO: #1}}
\newcommand{\martin}[1]{{\color{blue} Martin: #1}}
\newcommand{\simon}[1]{{\color{green} Simon: #1}}
\newcommand{\abcdef}[1]{{\color{red} Author2: #1}}
\newcommand{\rewrite}[1]{{\color{red} rewrite: #1}}
\newcommand{\ducky}[1]{{\color{orange} Richard: #1}}
\newcommand{\kasper}[1]{{\color{purple} Kasper: #1}}
\newcommand{\hjd}[1]{{\color{pink} Hans: #1}}

% uncomment following for final submission
%\renewcommand{\todo}[1]{}
%\renewcommand{\martin}[1]{}
%\renewcommand{\simon}[1]{}
%\renewcommand{\kasper}[1]{}
%\renewcommand{\ducky}[1]{}



%%% ZF
\usepackage{listings}
\lstset{
	columns=fullflexible,
	%        basicstyle=\ttfamily\footnotesize,
	basicstyle=\ttfamily\small,      
	%columns=fullflexible, keepspaces=true,
	numbers=left,    
	numberblanklines=false,
	captionpos=b,
	%	breaklines=true,
	escapeinside={@}{@},
	numbersep=5pt,
	language=C,
	tabsize=2,
	breakatwhitespace=true,
	breaklines=true,
	deletekeywords={for},
	%        keywordstyle=\ttfamily
	numbersep=5pt,
	xleftmargin=.10in,
	%xrightmargin=.25in
}

\newcommand{\longlist}[3]{{\lstinputlisting[float, caption={#2}, label={#3}, frame=tb, captionpos=b]{#1}}}

\title{Open-Source Chip Design}

%\author{
%\IEEEauthorblockN{No Author Given}
%\IEEEauthorblockA{No Institute Given}
%}

\author{\IEEEauthorblockN{XXX YYY\IEEEauthorrefmark{1}, \\
%Christa Skytte Jensen <christa@jensen.cc>
%Syed Anas Alam <s194016@student.dtu.dk>
%Nicolai B�low <nixen.dyre@gmail.com>
%Karl Herman Krause <s203852@student.dtu.dk>
%Andreas Alkj�r Eriksen <s203866@student.dtu.dk>
%Mads Rumle Nordstr�m <s200621@student.dtu.dk>
%Jakob Furbo Enevoldsen <s184074@student.dtu.dk>
%Niels William Hartmann <s203884@student.dtu.dk>
%Simon Winther Rasmussen <s203842@student.dtu.dk>
%Ulrik Helk <s203826@student.dtu.dk>
%Jonas Ingerslev S�rensen <s193274@student.dtu.dk>
%Tjark Petersen <s186083@student.dtu.dk>
%J�rgen Kragh Jakobsen <jkj@myrun.dk>
%Luca Pezzarossa <lpez@dtu.dk>
\\
xxx yyy\IEEEauthorrefmark{1}, zzz yyy\IEEEauthorrefmark{1}}\\
\IEEEauthorblockA{\IEEEauthorrefmark{1}\textit{Department of Applied Mathematics and Computer Science} \\
\textit{Technical University of Denmark}\\
Lyngby, Denmark \\\\
XXX, xxx@dtu..dk}
}


\begin{document}

%\IEEEoverridecommandlockouts
%\IEEEpubid{\makebox[\columnwidth]{xxx-\$31.00~\copyright2021 IEEE \hfill} \hspace{\columnsep}\makebox[\columnwidth]{ }}

\maketitle

\IEEEpubidadjcol

%\thispagestyle{empty}
\pagestyle{empty}

\begin{abstract}
Chip design is usually a high-effort, high-cost, and high-risk process. However, recently Google took the lead to
bring back chip design to a level where many engineers and student can participate, even enabling it for students.
In this paper we report from a course at the Technical University of Denmark where a group of students where able
to design and build a chip with the help of open-source tools and the SkyWater process.
\end{abstract}


\begin{IEEEkeywords}
digital design, verification, Chisel, Scala
\end{IEEEkeywords}

%\section{TODO}
%
%\begin{itemize}
%\item REVIEW 1: Re-write abstract and introduction - remove anything not related to verification. Remove negative wording on Verilog and VHDL. Re-consider necessity of section III.A and Fig. 2. Need for a more advanced use case.
%\item REVIEW 2: Include ChiselTest and ChiselVerify in Fig. 1. Re-write sections 4, 5 and 6, and ensure close connection to introduction. Add a few words on ChiselVerify as an agile development tool. Need for a more advanced use case (with comparison to UVM). Remove filler sentences.
%\item REVIEW 3: Prove claimed productivity increase.
%\item REVIEW 4: Need for a more advanced use case.
%\item REVIEW 5: Prove claimed productivity increase. Add table of available methods (section III).
%\end{itemize}

\section{Introduction}
\label{sec:introduction}

\emph{What this is about}

\todo{We aim for NorCAS. This is a 6 pages paper.}

\section{The Patmos Project}

\todo{Background info on the Patmos project that we use here. Martin shall write here.}

\section{The Tool Chain}

\todo{Background info on yosys,Google/SkyWater/efabless and other stuff}

\section{Booting Patmos}

The standard configuration of Patmos contains a Boot ROM that contains a program that downloads the real
program via a serial port. However, with the combination of the RISC-V processor we need to find a different way
to boot Patmos. %\martin{BTW, what is the argument here?}


\section{Hardware Design}

Although the Patmos/T-CREST project is a mature project that has been ported to many FPGA
boards, integrating it into the Caravel chip design flow created several challenges.

\subsection{External Memory}

The chip and PCB provided by the Caravel project makes only 38 \todo{check the number} pins available
for the project. This restriction makes it impossible to use standard SRAM or Flash devices. Therefore,
we needed to find a solution for external memories with a low pin count.

\todo{describe SPI RAM and Flash devices and the interface}

\subsection{Memory Compiler}


\subsection{Other}

Placeholder for better titles.


\subsection{xxx}
\subsection{xxx}

\section{Using Caravel}

\martin{This is probably not the right heading.}

\subsection{xxx}
\subsection{xxx}

\section{Results}

\martin{Will we have results?}

\section{Discussion}

\martin{As we do not have any chip results, we have a discussion section. However, we still can provide results
related to the process of doing a chip in an OS design flow.}

\section{Conclusion}
\end{document}